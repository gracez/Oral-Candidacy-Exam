\subsection{Intensity of Attraction}

Asymptotic resilience notably relies on linearizing at a point attractor. In contrast, intensity of attraction, introduced by Katherine Meyer in her PhD thesis \cite{meyerMetricPropertiesAttractors2019}, measures resilience not only for rest points but also for any other type of attractor. Even more importantly, it captures metric information across the entire basin of attraction rather than simplifying to a topologically equivalent approximation within a limited neighborhood. We now review the necessary background in order to define intensity of attraction. 

First of all, the idea of perturbation will now be represented by what is known as a \textbf{control function} added to an underlying vector field. This construction allows for perturbations which are not necessarily small and isolated, but possibly large and continuous, reflecting important types of perturbation that commonly occur in ecological and other applied settings, such as environmental forces or human-driven pressure (intentional or unintentional) on an ecosystem. We assume that the control function $$g: I \subset \mathbb{R} \to \mathbb{R}^n$$ is taken from the space of essentially bounded (i.e. bounded except on a set of measure 0) measurable functions $L^\infty (I,\mathbb{R}^n)$, where the norm is 
$$||g||_\infty = \inf\{C \geq 0  :  ||g(x)|| \leq C  \text{ for almost every } x \in I \}.$$ 
We also assume $g$ is locally integrable (i.e. integrable on every compact subset of its domain $I$). These mild assumptions will ensure that $g$ is nice enough for our perturbed system to remain well-defined. Next, we formalize how the perturbation is added to an underlying system. 

\begin{definition}
	A \textbf{bounded control system} \todo{is this ok terminology?} is a non-autonomous ODE 
	\begin{equation}
		\label{eqn:control_ode}x' = f(x) + g(t)
	\end{equation}
	where $f: U \subset \mathbb{R}^n \to \mathbb{R}^n$ is locally Lipschitz, $g \in L^\infty (I,\mathbb{R}^n)$ is locally integrable, and its norm $||g||_\infty$ is finite. \todo{how can $g$ have infinite norm and be locally integrable? do i need this as a separate condition?} \qed
\end{definition}

Here, the idea is that there is an underlying ODE $x'=f(x)$, but it is altered by adding the perturbation $g(t)$ to the vector field $f(x)$ on the right hand side. At every point in time, $g(t)$ will adjust the path of solutions somewhat away from their original trajectory. 

It remains to be justified whether such an addition leaves a well-defined system. Because the right hand side $f(x) + g(t)$ may be a discontinuous function, solutions $x(t)$ of the ODE must be considered in an extended sense, which is that $x'(t) = f(x) + g(t) \text{ almost everywhere.}$ Fortunately, the conditions on $g$ are enough to guarantee local existence and uniqueness of solutions in such an extended sense. Briefly: (1) the hypotheses of Caratheodory's theorem are satisfied, establishing existence, and (2) boundedness of $g$ guarantees Lipschitz continuity (local if $f$ is locally Lipschitz, global if $f$ is globally Lipschitz), thereby implying uniqueness. 

So we have well-defined solutions, and can therefore extend the standard local flow notation to the bounded control setting. Fixing an underlying vector field $f$, we will denote as follows the flow obtained by applying a choice of perturbation $g$.

\begin{definition} 
	$\varphi_g(t, x_0): D \subset \mathbb{R} \times U \to U$ is the local flow defined by $$\varphi_g(t, x_0) = x(t)$$ where $x(t)$ solves in the extended sense the ODE (\ref{eqn:control_ode}), with initial condition $x(0) = x_0$. \qed
\end{definition}

Intensity of attraction considers not just one single control function, but entire families of control functions -- specifically, those where every function is bounded by some maximum magnitude $r$. Supposing that vector field perturbations are limited by some ceiling $r$ on magnitude, each family can be thought of as a collection of all possible perturbations. The next definition gives a notation, $B_r$ for these families.


\begin{definition}
	Denote by $B_r \subset L^\infty[I, \mathbb{R}^n]$ the set of control functions bounded above by $r$:
	$$B_r = \{g  : ||g||_\infty < r\}$$ \qed
\end{definition}

%Then, again fixing an underlying vector field $f$, the collection of all possible perturbed trajectories can be captured by the following notation.

%\begin{definition} 
%	\begin{align*}
%		\Psi_{r} = \{ 
%		(a, b, T) : ~&\exists
%		\text{ a control function } 
%		g \in B_r \text{ such that some solution } x:[0,T] \to \mathbb{R}\\
%		& \text{ of the associated bounded control system }
%		x' = f(x) + g(t)\\
%		& \text{ has endpoints }
%		x(0) = a, x(T) = b
%		\}
%	\end{align*} \qed
%\end{definition}

Next, this leads into the notion of all possible states reachable in forward time, under control bounded by $r$, and beginning from some arbitrary initial set. 

\begin{definition}
	Consider $S\subset  U$. The \textbf{reachable set} of $S$ under $r$-bounded control is
	$$R_r(S) =  \bigcup\limits_{g \in B_r} \bigcup\limits_{x_0 \in S} \bigcup\limits_{t \geq 0}  \varphi_g(t,x_0)$$ \qed
\end{definition}

Finally, we are ready to define intensity of attraction, which captures the folllowing idea: what is the smallest magnitude of control necessary in order to escape from a basin of attraction? \todo{why compact subsets?}

\begin{definition}
	If $A$ is an attractor of $x' = f(x)$, then its \textbf{intensity of attraction} is 
	$$intensity(A) = \sup\{ r \geq 0 ~|~ R_r(A) \subset K \subset basin(A), \text{ for some compact }K \}$$ \qed
\end{definition}

\todo{add example}