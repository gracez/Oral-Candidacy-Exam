
\section{Introduction}
\label{sec:intro}

A \textbf{tipping point} or \textbf{critical transition} occurs in a dynamical system when a small perturbation to system conditions causes an abrupt overall shift in qualitative behavior. 
%
Empirically, tipping points have been studied in contexts as diverse as 
Earth's climate \cite{lentonTippingElementsEarth2008, dakosSlowingEarlyWarning2008a}, 
emerging infectious diseases \cite{brettDynamicalFootprintsEnable2020}, 
aquatic and land ecosystems \cite{schefferCatastrophicShiftsEcosystems2001a, carpenterRisingVarianceLeading2006}, 
the onset of medical health states \cite{mcsharryPredictionEpilepticSeizures2003, venegasSelforganizedPatchinessAsthma2005}, 
financial markets \cite{gatfaouiFlickeringInformationSpreading2019},
and more \cite{georgeEarlyWarningSignals2021, schefferEarlywarningSignalsCritical2009a, boettigerEarlyWarningSignals2013}. 
%
Since critical transitions often represent a shift into an undesirable or catastrophic regime, and since such transitions may not be easily or at all reversible, %\cite{albrichClimateChangeCauses2020, chenImperfectVaccineHysteresis2019, lucariniThermodynamicAnalysisSnowball2010}
it is of pressing interest to anticipate them before they occur, in order to inform management strategies and improve the odds of prevention. Unfortunately, in complex real world systems, the conditions under which a critical transition occurs, and the underlying mechanisms driving the approach to transition, are usually extremely difficult to characterize.

As a result, there is particular interest in generic mathematical signals that can warn of impending tipping in a wide variety of systems without reference to specific underlying mechanisms. Such \textbf{early warning signals} have been most commonly studied as precursors of local codimension-1 bifurcations of ODEs, where they are based on the phenomenon of \textbf{critical slowing down} \cite{schefferEarlywarningSignalsCritical2009a}. Roughly speaking, as the bifurcation parameter gradually nears its critical value, the resilience of the system drops (becoming slower to recover from perturbations), and this produces certain detectable statistical trends over time. In the context of critical slowing down, the term resilience refers specifically to what is known in the ecology literature as \textbf{asymptotic resilience}. In Section (\ref{sec:resilience}), I review two different quantifications of resilience, asymptotic resilience and another known as \textbf{intensity of attraction}). In Section (\ref{sec:csd}), I summarize the theory of critical slowing down. 

Early warning signals derived from asymptotic resilience and critical slowing down are a powerful tool for anticipating critical transitions, and their usefulness has already been demonstrated in numerous empirical contexts including, for instance, early detection of emerging infectious diseases \cite{brettDynamicalFootprintsEnable2020} and retrospective analyses of prehistoric climate change events \cite{dakosSlowingEarlyWarning2008a}. But a major simplification is the assumption that the system experiences only small, infrequent perturbations, which do not drive the system state very far from equilibrium and which leave sufficient time for recovery in between disturbances. In particular, there is a neglect of transient behavior within the larger domain of attraction. Such transient states can result from large, closely repeated, or continual disturbances, as are common in real world systems. A second shortcoming arises from the fact that critical slowing down relates only to one specific category of tipping behavior -- local bifurcations. In contrast, another way a dynamical system might tip is due to perturbations pushing the system state across a boundary between domains of alternative attractors. (Many other dynamical behaviors also correspond to tipping, and are not considered here, including global bifurcations, rate-induced tipping, and transitions to chaotic regimes.)

%Early warning signals derived from certain transient dynamics have been developed recently in the infectious disease literature \cite{oreganTransientIndicatorsTipping2020}. In Section (\ref{sec:transient}) I first review transient indicators arising from reactivity. 

Early warning signals derived from transient dynamics are a research area that demand future development. In Section (\ref{sec:proposal}), the thesis proposal, I consider the possibility for transient indicators to arise from intensity of attraction, and suggest preliminary steps toward developing an understanding of such indicators. %systematic changes in intensity through critical transitions. 








%Kate: 
%"Ecological modelers commonly quantify resilience—roughly, a system’s capacity to absorb change and disturbance while maintaining its basic structure and function [31]—in terms of a system’s invariant sets. Some use eigenvalues of linearization at an attracting equilibrium, which measure resilience to small perturba- tions in terms of characteristic return times [23]. Others, concerned with resilience to potentially large perturbations, measure the size of a domain of attraction [12], e.g. via n-dimensional domain volume [18] or distance from attractor to domain boundary [2]. Both eigenvalues and domain size lack information regarding transient dynamics within a domain of attraction, so neither fully reveals how a system will respond to continuous disturbances that drive the state away from invariant sets."















