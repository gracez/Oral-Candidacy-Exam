\section{Introduction}
\label{sec:intro}

\subsection{Critical Transitions}

A tipping point or critical transition occurs in a dynamical system when a small perturbation to the system causes an abrupt qualitative shift in overall behavior. This informal concept is often understood as a local bifurcation, but may also correspond to a variety of other dynamical behaviors including global bifurcations, perturbations pushing a state variable across the boundary between two basins of attraction, and rate-induced tipping. 

Empirically, critical transitions have been studied in contexts ranging from Earth's climate \cite{lentonEarlyWarningClimate2011} to emerging infectious disease \cite{brettDynamicalFootprintsEnable2020} to


In complex empirical systems, the conditions under which a critical transition occurs are generally extremely difficult to predict. In many cases, the underlying mechanisms driving such a system toward the brink may be impossible to fully understand or identify. 

\subsection{Motivation}


\cite{george_early_2021}