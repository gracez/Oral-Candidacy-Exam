\section{Introduction}
\label{sec:intro}

A tipping point or critical transition occurs in a dynamical system when a small perturbation of conditions causes an abrupt qualitative shift in overall system behavior. This informal concept is often understood as a local bifurcation, but may also correspond to a variety of other dynamical behaviors such as global bifurcations, perturbations pushing a state variable across the boundary between two basins of attraction, or rate-induced tipping. 

Critical transitions have been studied in empirical contexts ranging from Earth's climate \cite{lentonTippingElementsEarth2008} to ecological systems \cite{schefferCatastrophicRegimeShifts2003} to emerging infectious disease \cite{brettDynamicalFootprintsEnable2020}. 

Hard to 


In complex empirical systems, the conditions under which a critical transition occurs are generally extremely difficult to predict. In many cases, the underlying mechanisms driving such a system toward the brink may be impossible to fully understand or identify. 

