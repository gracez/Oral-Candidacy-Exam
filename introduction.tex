\section{Introduction}
\label{sec:intro}

A \textbf{tipping point} or \textbf{critical transition} occurs in a dynamical system when a small perturbation to system conditions causes an abrupt overall shift in qualitative behavior. This informal concept is often understood as a local bifurcation in a dynamical system, but may also correspond to other dynamical behaviors such as a global bifurcation, a perturbation pushing a state variable across the boundary between two basins of attraction, or rate-induced tipping behavior. 

Empirically, tipping points have been studied in a wide range of contexts, including Earth's climate \cite{lentonTippingElementsEarth2008, dakosSlowingEarlyWarning2008a}, emerging infectious diseases \cite{brettDynamicalFootprintsEnable2020}, aquatic and land ecosystems \cite{schefferCatastrophicShiftsEcosystems2001a, carpenterRisingVarianceLeading2006}, the onset of medical health conditions \cite{mcsharryPredictionEpilepticSeizures2003, venegasSelforganizedPatchinessAsthma2005}, and socio-economic systems \cite{ginkelClimateChangeInduced2020}, among others \cite{georgeEarlyWarningSignals2021, schefferEarlywarningSignalsCritical2009a}. Since critical transitions often represent a shift into an undesirable or catastrophic regime, and since such transitions may not be easily or at all reversible \cite{albrichClimateChangeCauses2020, chenImperfectVaccineHysteresis2019, lucariniThermodynamicAnalysisSnowball2010}, it is of utmost interest to try to anticipate them before they occur, in order to improve the odds of prevention. For instance, the ever-growing web of complex physical, ecological, and social pressures exerted by modern day anthropogenic practices on the Earth threaten planetary, human, and animal systems with unprecedented and potentially disastrous forms of change. 

In real world systems, the conditions under which a critical transition occurs, and the underlying mechanisms driving the approach to transition can be extremely difficult to characterize; hence, there is particular interest in identifying generic mathematical signals that can warn of impending tipping in a wide variety of systems regardless of specific underlying mechanisms. Such \textbf{early warning signals} have been identified in the context of local bifurcations of ODEs, and are based on the phenomenon of \textbf{critical slowing down} \cite{schefferEarlywarningSignalsCritical2009a}: roughly, as a stable rest point approaches a critical loss of stability, the resilience of the system gradually decreases. This leads to certain detectable statistical trends over time - in particular, resulting in increasing variance and autocorrelation in the system state. 




