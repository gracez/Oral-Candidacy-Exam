\section{Introduction}
\label{sec:intro}

A \textbf{tipping point} or \textbf{critical transition} occurs in a dynamical system when a small perturbation to system conditions causes an abrupt overall shift in qualitative behavior. This informal concept is often understood as a local bifurcation in a dynamical system, but may also correspond to other dynamical behaviors such as global bifurcations, perturbations pushing a state variable across the boundary between two basins of attraction, or rate-induced tipping behavior. 

Empirically, tipping points have been studied in a wide range of contexts, including 
%
Earth's climate \cite{lentonTippingElementsEarth2008, dakosSlowingEarlyWarning2008a}, 
%
emerging infectious diseases \cite{brettDynamicalFootprintsEnable2020}, 
%
aquatic and land ecosystems \cite{schefferCatastrophicShiftsEcosystems2001a, carpenterRisingVarianceLeading2006}, 
%
the onset of medical health conditions \cite{mcsharryPredictionEpilepticSeizures2003, venegasSelforganizedPatchinessAsthma2005}, 
%
socio-economic systems \cite{ginkelClimateChangeInduced2020}, 
%
and more \cite{georgeEarlyWarningSignals2021, schefferEarlywarningSignalsCritical2009a}. 
%
%
Since critical transitions often represent a shift into an undesirable or catastrophic regime, and since such transitions may not be easily or at all reversible \cite{albrichClimateChangeCauses2020, chenImperfectVaccineHysteresis2019, lucariniThermodynamicAnalysisSnowball2010}, it is of utmost interest to anticipate them before they occur, in order to improve the odds of prevention. 
%
As pressures exerted by modern day anthropogenic practices on the Earth grow in magnitude and complexity, threatening physical, ecological, and social systems on all scales with unprecedented forms of change, this goal becomes even more pressing. 
%
Unfortunately, in complex real world systems, the conditions under which a critical transition occurs, and the underlying mechanisms driving the approach to transition are usually extremely difficult to characterize.

Hence, there is particular interest in generic mathematical signals that can warn of impending tipping in a wide variety of systems without reference to specific underlying mechanisms. 
%
Such \textbf{early warning signals} have been most commonly studied as precursors of local codimension-1 bifurcations of ODEs, where they are based on the phenomenon of \textbf{critical slowing down} which occurs as the bifurcation parameter gradually nears its critical value \cite{schefferEarlywarningSignalsCritical2009a}.
% 
Roughly, the resilience of the system gradually drops, and this results in certain detectable statistical trends over time -- in particular, gradually increasing variance and auto-correlation in the system state. In the context of local codimension-1 bifurcations and critical slowing down, the term resilience refers to the dominant eigenvalue of the Jacobian at the stable rest point. The real part of this eigenvalue approximately governs the rate of return to equilibrium after a small perturbation to the system. Because local bifurcation is characterized by the real part of the dominant eigenvalue passing through zero, the system recovers slower when nearer to bifurcation. This definition of resilience (dominant eigenvalue of the Jacobian) is also known in the ecology(?) literature as \textbf{asymptotic resilience}. 










