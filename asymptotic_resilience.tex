\subsection{Asymptotic Resilience}

Throughout this subsection, we will assume that $x_\ast$ is an attracting rest point of an ODE. Probably the most commonly used mathematical definition of resilience,  originally developed by theoretical ecologists\todo{citation}, represents long-term return rates to $x_{\ast}$, and is measured by (the real part of) the dominant eigenvalue at linearization. 

\begin{definition}
	\label{def:asymp}
	 Let $\textbf{A} = Df(x_\ast)$ denote the Jacobian, and recall that all eigenvalues of $\mathbf{A}$ have negative real part. Let $\lambda_1(\textbf{A})$ be the eigenvalue with largest (closest to 0) real part. 
	
	\begin{center}
	The \textbf{asymptotic resilience} of the system at the stable rest point is $Re(\lambda_1(\textbf{A}))$.
	\end{center}
Note: we will refer to $\lambda_1$ as the \textbf{dominant eigenvalue} of $\mathbf{A}$. 
	 \qed 
\end{definition}

For the linearized system $x'= \textbf{A}x$, asymptotic resilience estimates the rate at which trajectories approach the equilibrium. The following theorem is standard theory for linear ODEs. %See for example (Chicone p. 175) \todo{do citation}


\begin{theorem}
	For an $n \times n$ matrix $\mathbf{A}$, if $Re(\lambda) < L < 0$ for all eigenvalues $\lambda$ of $\mathbf{A}$, then there is some constant $C>0$ such that for all $x \in \mathbb{R}^n$ and $t \geq 0$,
	
	$$|e^{tA}x| \leq Ce^{L t}|x|.$$ 
	
	Further, in the long term $C$ can be taken to equal $1$. That is, there is some $T \geq 0$ such that
	
	$$|e^{tA}x| \leq e^{L t}|x| ~ ~\text{ for all } t \geq T.$$ \qed
\end{theorem}

%\begin{theorem}
%	For an $n \times n$ matrix $\mathbf{A}$, if $Re(\lambda) < L < 0$ for all eigenvalues $\lambda$ of $\mathbf{A}$, then there is some constant $T>0$ such that for all $x \in \mathbb{R}^n$ and $t \geq T$,
%	
%	$$|e^{tA}x| \leq e^{L t}|x|.$$  \qed
%\end{theorem}

Note the expression $e^{tA}x$ in the left hand side is exactly the flow for the linear system $x' = \mathbf{A}x$. So this theorem says that, in the long term, trajectories must decay to the equilibrium at the origin at an exponential rate. Further, a bound on that rate is governed by the asymptotic resilience. 

For nonlinear systems, a similar bound on decay rate is justified by the Stable Manifold Theorem, a fundamental result in dynamical systems theory which says that, at sufficiently nice rest points, the linear approximation is a good approximation. %, also a standard theoretical result. 
%
%The theorem states that there is a local  Moreover, the flow restricted to the stable and the unstable manifolds has exponential (hyperbolic) estimates similar to the inequalities in display (4.1)
%
%
A special case of the Stable Manifold Theorem is stated here, while a full version can be found in any standard text. %is relegated to the appendix. 

\begin{theorem}(Stable Manifold Theorem, for attracting rest points)
	Consider a non-linear system 
	%
	$$x' = \mathbf{A}(x) + h(x),$$ 
	%
	where $\mathbf{A}, h: \mathbb{R}^n \to \mathbb{R}^n$ with $\mathbf{A}$ linear.  Let $\phi_t$ be the local flow.
	%
	Assume there is an attracting rest point at the origin. 
	%
	Let $\lambda_1$ be the dominant eigenvalue of $\mathbf{A}$. Then there exists a neighborhood $N \ni 0$ which is a \textbf{local stable manifold} of the origin. 
	%
	That is, for all $x \in N$, $\lim\limits_{t \to \infty} \phi_t(x)= 0$.
	
	Furthermore, for any $Re(\lambda_1) < L < 0$, there exists $C >0$ such that for all $x \in N$, $t \geq 0$,
	%
	$$|\phi_t(x)| \leq Ce^{Lt}|x|,$$
	%
 	and for some $T \geq 0$, $C$ can be taken to equal $1$
	%
	$$|\phi_t(x)| \leq e^{L t}|x| ~ ~\text{ for } t \geq T.$$
	%
	\todo{does the long term $C=1$ statement hold? Can't find this in any book but I think it should be true} \qed
\end{theorem}

The bound on decay rate stated in the last line of the theorem implies that any trajectory beginning sufficiently close to equilibrium decays toward equilibrium at an exponential rate, and a long term bound on that rate is governed by the asymptotic resilience. Since the equilibrium represents a steady state, then any point which is very close to, but not quite at, the equilibrium represents a slightly perturbed state. Hence, the rate of decay represents the recovery rate from perturbations. 

%Hence, asymptotic resilience bounds the rate of return to equilibrium after a small perturbation to the system. Because local bifurcation is characterized by $Re(\lambda_1)$ passing through zero, the system recovers slower when nearer to bifurcation. This is the core idea of critical slowing down, which will be explained further in Section \ref{sec:csd}.

\begin{remark}
	Note that trajectories need not decay monotonically in distance to the rest point, not even for linear systems. For instance, a trajectory can initially amplify in magnitude -- a phenomenon termed \textbf{reactivity} by Neubert and Caswell in \cite{neubertAlternativesResilienceMeasuring1997a} (Figure \ref{fig:reactivity}). However, with some large enough choice of $T$, the Stable Manifold Theorem implies that this short term growth negligibly affects the long term decay. 
\end{remark}

\begin{figure}[ht]
	\centering
	\captionsetup{width=0.8\linewidth}
	\includegraphics[width=0.8\textwidth]{figs/positive_reactivity_real_example}
	\caption{Phase portraits of two linear systems $x' = \textbf{A}x$. (a) All trajectories decay monotonically in magnitude. (b) There are trajectories beginning arbitrarily close to the origin which initially increase in magnitude. Notice that both matrices have the same eigenvalues $\lambda = -1, -2$; hence asymptotic resilience cannot tell whether an equilibrium is reactive. Example reproduced from \cite{neubertAlternativesResilienceMeasuring1997a}.}
	
	\label{fig:reactivity}
\end{figure} 