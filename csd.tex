\section{Critical Slowing Down}
\label{sec:csd}

this results in certain detectable statistical trends over time -- in particular, gradually increasing variance and auto-correlation in the system state

\subsection{Local Bifurcation}

\subsection{Critical Slowing Down}

\subsection{Early Warning Signals}

\subsection{Limitations}
	
Early warning signals derived from critical slowing down are a powerful tool for anticipating critical transitions, and their usefulness has already been demonstrated in numerous empirical contexts, including \todo{citations}. However, they have at least a few significant limitations. 
%
First, being based on a linear approximation at a stable equilibrium, they are relevant only to small perturbations that do not move the system state very far from equilibrium. 
%
Second, being a measure of long term rates of return to equilibrium, they (1) may overlook short term behavior that occurs immediately after the perturbation and (2) are relevant only to infrequent perturbations, so that the system has enough time to recover in between disturbances. In particular, they are not reliable in cases of closely repeating or continual disturbances, as are common in real world ecological systems. 
%
Third, they specifically precede local bifurcations, while the informal tipping point concept may correspond to other dynamical behaviors such as global bifurcations, perturbations pushing a state variable across the boundary between two basins of attraction, or rate-induced tipping behavior. 