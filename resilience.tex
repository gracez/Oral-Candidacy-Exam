\section{Resilience Quantification}
\label{sec:resilience}

Loosely, resilience refers to the capacity for a system to retain its overall qualitative structure in the face of disturbances. Its precise definition differs between authors and disciplines; an abundance of approaches to quantifying resilience have been proposed. 
%
In this section, I define asymptotic resilience and intensity of attraction. 
%
For a review of other mathematical definitions of resilience that I do not cover, see \cite{meyerMathematicalReviewResilience2016}.


\subsection{Preliminaries}
Let $U \subset \mathbb{R}^n$ be open, and assume that $f : U \to \mathbb{R}^n$ is locally Lipschitz continuous. Consider a system of ODEs 

\begin{equation}
	\label{eqn:ode}
	x' = f(x)
\end{equation}

and let $\varphi: D \subset \mathbb{R} \times U \to U$ be the associated local flow, so that $\varphi(t,x_0) = x(t)$ solves the ODE (i.e. $x'(t) = f(x(t))$), with initial condition $x(0) = x_0$. 

In many contexts, $f$ is globally Lipschitz continuous, in which case trajectories are defined for as long as they remain within $U$. If no trajectories leave $U$, then $\varphi$ is a global flow defined for all $t \in \mathbb{R}$. In this paper we will assume for the sake of simplicity that the flow is defined on any time domain of interest. 

There are also a couple of notational conveniences we will take. For a flow $\varphi: \mathbb{R} \times U \to U$, we will denote the time $t$ map as $\varphi_t: U \to U$. We will also naturally extend its definition to allow set-valued inputs $S\subset U$ as follows: $$\varphi_t(S) = \{x\in U ~| ~\varphi_t(x_0) = x \text{ for some } x_0 \in S\}.$$

Two central objects of study in this paper is are attractors and their associated domains, or basins of attraction. An attractor characterizes the eventual behavior approached by the system over time (or at least the part of the system lying within its basin).
% Tipping comes down to either an abrupt shift in the nature of the attractor or an abrupt shift to an alternative attractor. 
In order to define attractors and basins, we must first formalize some aspects of long-term behavior.

\begin{definition}
	Consider a subset $S \subset U$. $S$ is \textbf{forward invariant} under the flow $\varphi$ if it contains all its forward images in time: $\varphi_t(S) \subset S$ for all $t \in \mathbb{R}^+$. Similarly, $S$ is \textbf{backward invariant} if $\varphi_t(S) \subset S$ for all $t \in \mathbb{R}^-$. $S$ is \textbf{invariant} if it is both forward and backward invariant. \qed
\end{definition}

Intuitively, an invariant set is one which always stays within itself. The next definition describes where an arbitrary set limits toward in the long run. 

\begin{definition}
	The \textbf{omega limit set} of $S \subset U$ is $$\omega(S)
	= \bigcap \limits_{T > 0} \overline{ \bigcup\limits_{t > T} \varphi_t(S) }.$$ \qed
\end{definition}

%The omega limit set describes where trajectories that begin in $S$ eventually limit toward in forward time. Now we are ready to define an attractor. Additionally, an attractor has an associated domain, or basin of attraction, which is the subspace attracted to it. 

Now we can define an attractor and its basin.

\begin{definition}
	An \textbf{attractor} $A \subset U$ is a non-empty, compact, invariant set which is the omega limit set $\omega(N)$ of some neighborhood $N$ of itself. Its \textbf{basin of attraction} is $basin(A) = \{x \in U ~|~ \omega(x) \subset A, \omega(x) \neq \emptyset\}$. \qed
\end{definition}

While attractors may have interesting structures -- periodic or chaotic, for instance -- we will begin with the simplest type of attractor: an attracting rest point. 

\begin{definition}
	$x_\ast$ is a \textbf{rest point} or \textbf{equilibrium} of the ODE (\ref{eqn:ode}) if $f(x_\ast) = 0$.
\end{definition}

%\begin{definition}
	%A rest point $x_\ast$ is a \textbf{asymptotically stable} if
%\end{definition}

The following proposition is standard theory. 

\begin{proposition}
	If all eigenvalues of linearization at the rest point $x_{\ast}$ have negative real part, that is, $$Re(\lambda) < 0 \text{ for all } \lambda \in spec(\textbf{D}f(x_\ast))$$ 
	then $x_{\ast}$ is an attractor. Note: we also call such an $x_\ast$ a \textbf{stable rest point}. \todo{what is the best terminology here?}
\end{proposition}

\subsection{Asymptotic Resilience}
\label{sec:asymp_res}

Throughout this subsection, we will assume that $x_\ast$ is an attracting rest point of a continuously differentiable ODE. Probably the most commonly used mathematical definition of resilience,  originating in theoretical ecology \cite{pimmComplexityStabilityEcosystems1984, mayStabilityComplexityModel1974, hollingResilienceStabilityEcological1973, pimm1991balance}, represents long-term return rates to $x_{\ast}$, and is measured by (the real part of) the dominant eigenvalue at linearization. 

\begin{definition}
	\label{def:asymp}
	 Let $\textbf{A} = Df(x_\ast)$ denote the Jacobian, and recall that all eigenvalues of $\mathbf{A}$ have negative real part. Let $\lambda_1(\textbf{A})$ be the eigenvalue with largest (closest to 0) real part. 	The \textbf{asymptotic resilience} of the system at the stable rest point is equal to the negative of that real part, $$-Re(\lambda_1(\textbf{A})).$$

Note: we will refer to $\lambda_1$ as the \textbf{dominant eigenvalue} or the \textbf{slow eigenvalue} of $\mathbf{A}$. 
	 \qed 
\end{definition}

For the linearized system $x'= \textbf{A}x$, asymptotic resilience estimates the rate at which trajectories approach the equilibrium. The following theorem is standard theory for linear ODEs. %See for example (Chicone p. 175) \todo{do citation}


\begin{theorem}
	For an $n \times n$ matrix $\mathbf{A}$, if $Re(\lambda) < L < 0$ for all eigenvalues $\lambda$ of $\mathbf{A}$, then there is some constant $C>0$ such that for all $x \in \mathbb{R}^n$ and $t \geq 0$,
	%
	$$|e^{t\mathbf{A}}x| \leq Ce^{L t}|x|.$$ 
	%
	And in the long term $C$ can be taken to equal $1$. That is, there is some $T \geq 0$ such that
	%
	$$|e^{t\mathbf{A}}x| \leq e^{L t}|x| ~ ~\text{ for all } t \geq T.$$
	
	%Further, in the limit, and for all $x$ except on a set of measure 0, the inequality can be replaced with equality and $L$ can be replaced with asymptotic resilience.
	%
	%$$\lim\limits_{t \to \infty} |e^{tA}x| = e^{Re(\lambda_1) t}|x| ~ ~\text{ for almost all } x.$$ \todo{make sure this last part is true and include a proof in the appendix}
	
	 \qed
\end{theorem}


%TO DO: Is there a converse to the inequality? how to say that this is a good bound? i.e. for almost all trajectories, and in the limit as $t\to \infty$, they do eventually decay at that rate, rather than much faster than it. I feel like this is true, but I can't find a statement of it in a book. \todo{to do}

%\begin{theorem}
%	For an $n \times n$ matrix $\mathbf{A}$, if $Re(\lambda) < L < 0$ for all eigenvalues $\lambda$ of $\mathbf{A}$, then there is some constant $T>0$ such that for all $x \in \mathbb{R}^n$ and $t \geq T$,
%	
%	$$|e^{tA}x| \leq e^{L t}|x|.$$  \qed
%\end{theorem}

Note the operator $e^{t\mathbf{A}}$ in the left hand side is exactly the flow $\varphi_t$ for the linear system $x' = \mathbf{A}x$. So this theorem says that, in the long term, trajectories must decay to the origin at an exponential rate governed by the asymptotic resilience. 

%A related theorem serves as a converse of sorts, and says that the exponential bound is a good bound, so that trajectories typically do not decay significantly faster. A proof is contained in the appendix.  \todo{Is this correct? Need to work through a proof carefully. I can't find this in a book.}
%
%\begin{theorem} For almost all initial conditions $x\in U$, and in the limit as $t \to \infty$,
%	$$ |e^{tA}x| = e^{ Re(\lambda_1)t}|x|.$$ \qed
%\end{theorem}

%\begin{theorem}
%	For almost all initial conditions $x_0 \in U$, 
%	$$\lim\limits_{t \to \infty} |\varphi_t(x_0)|' = Re(\lambda_1)|\varphi_t(x_0)|,$$
%	where $' = \dfrac{d}{dt}$, and $\lambda_1$ is the dominant eigenvalue of $\mathbf{A}$.
%\end{theorem}

%\todo{write proof.}

For nonlinear systems, similar results for decay rate are justified by the Stable Manifold Theorem, a fundamental result which says that, at sufficiently nice rest points, the linear approximation is a good approximation. 
%
%The theorem states that there is a local  Moreover, the flow restricted to the stable and the unstable manifolds has exponential (hyperbolic) estimates similar to the inequalities in display (4.1)
%
%
A special case of the Stable Manifold Theorem is stated here, while a full version can be found in any standard text. %is relegated to the appendix. 

\begin{theorem}(Stable Manifold Theorem, for attracting rest points)
	Consider a non-linear system 
	%
	$$x' = \mathbf{A}(x) + h(x),$$ 
	%
	where $\mathbf{A}, h: \mathbb{R}^n \to \mathbb{R}^n$ with $\mathbf{A}$ linear.  Let $\phi_t$ be the local flow.
	%
	Assume there is an attracting rest point at the origin. 
	%
	Let $\lambda_1$ be the dominant eigenvalue of $\mathbf{A}$. Then there exists a neighborhood $N \ni 0$ which is a \textbf{local stable manifold} of the origin. 
	%
	That is, for all $x \in N$, $\lim\limits_{t \to \infty} \phi_t(x)= 0$.
	
	Furthermore, for any $Re(\lambda_1) < L < 0$, there exists $C >0$ such that for all $x \in N$, $t \geq 0$,
	%
	$$|\phi_t(x)| \leq Ce^{Lt}|x|,$$
	%
 	and for some $T \geq 0$, $C$ can be taken to equal $1$
	%
	$$|\phi_t(x)| \leq e^{L t}|x| ~ ~\text{ for } t \geq T.$$
	%Finally, in the limit, and for all $x\in N$ except on a set of measure 0, the inequality can be replaced with equality and $L$ can be replaced with asymptotic resilience.
	%
	%$$\lim\limits_{t \to \infty} \phi_t(x)| = e^{Re(\lambda_1) t}|x| ~ ~\text{ for almost all } x.$$ \todo{make sure this last part is true and include a proof in the appendix}
	\qed
\end{theorem}

%TO DO: does that last statement about long term $C=1$ hold? Can't find this in a book but I feel like it should be true. And also, what about a converse to the inequality? Does anything like that exist for the stable manifold theorem, if it does for linear systems? \todo{to do}

The theorem implies that any trajectory beginning sufficiently close to equilibrium decays toward equilibrium at an exponential rate, where that rate is determined in the long term by asymptotic resilience. Any point which is very close to, but not quite at, the equilibrium represents a state slightly perturbed away from steady state. Hence, the rate of decay can be thought of as the recovery rate from a small perturbation. 

%Hence, asymptotic resilience bounds the rate of return to equilibrium after a small perturbation to the system. Because local bifurcation is characterized by $Re(\lambda_1)$ passing through zero, the system recovers slower when nearer to bifurcation. This is the core idea of critical slowing down, which will be explained further in Section \ref{sec:csd}.

\begin{remark}
	Note that trajectories need not decay monotonically in distance to the rest point, not even for linear systems. For instance, a trajectory can initially amplify in magnitude -- a phenomenon termed \textbf{reactivity} by Neubert and Caswell in \cite{neubertAlternativesResilienceMeasuring1997} (Figure \ref{fig:reactivity}). However, with some large enough choice of $T$, the Stable Manifold Theorem still implies that short term growth negligibly affects long term decay. 
\end{remark}	

\begin{figure}[ht]
	\centering
	\captionsetup{width=0.8\linewidth}
	\includegraphics[width=0.8\textwidth]{figs/positive_reactivity_real_example}
	\caption{Phase portraits of two linear systems $x' = \textbf{A}x$. (a) All trajectories decay monotonically in magnitude. (b) There are trajectories beginning arbitrarily close to the origin which initially increase in magnitude. Notice that both matrices have the same eigenvalues $\lambda = -1, -2$; hence asymptotic resilience cannot tell whether an equilibrium is reactive. Example reproduced from \cite{neubertAlternativesResilienceMeasuring1997}.}
	
	\label{fig:reactivity}
\end{figure} 

%\subsection{Reactivity}
Asymptotic resilience governs the long-term rate of recovery. However, in the short term, perturbations can initially be amplified before eventually decaying to the stable equilibrium (Figure \ref{fig:reactivity}). Motivated by this transient behavior, an alternative measure of system response to perturbations was introduced by Neubert and Caswell in \cite{neubertAlternativesResilienceMeasuring1997a}.

\begin{figure}[ht]
	\centering
	\captionsetup{width=0.8\linewidth}
	\includegraphics[width=0.8\textwidth]{figs/positive_reactivity_real_example}
	\caption{Phase portraits of two linear systems $x' = \textbf{A}x$ with the same eigenvalues  $\lambda = -1, -2$. (a) All trajectories decay monotonically in magnitude. (b) Trajectories may transiently increase in magnitude before decreasing. Example reproduced from \cite{neubertAlternativesResilienceMeasuring1997a}.}
	
	\label{fig:reactivity}
\end{figure} 

\begin{definition}
	Suppose that $x_\ast$ is a stable rest point of the ODE (\ref{eqn:ode}). Let $\textbf{A} = Df(x_\ast)$ be the Jacobian, and let $\textbf{H} = \frac{\textbf{A}+\textbf{A}^T}{2}$ be its symmetric part. Since $\textbf{H}$ is a real symmetric matrix, it has real eigenvalues. Let $\lambda_1(\textbf{H})$ be the maximum eigenvalue.
	
	 \begin{center}
	 	The \textbf{reactivity} of the system at the stable rest point is $\lambda_1(\mathbf{H})$.\\ If this number is positive, the system is called \textbf{reactive}.
	 \end{center}
	
	\qed
\end{definition}

Reactivity measures the maximum possible relative rate of initial amplification. The following proposition, which relies on elementary linear algebra, shows this result for linear systems. 

\begin{proposition}[Neubert and Caswell]
	For the linear system $x' = \mathbf{A}x$, $\lambda_1(\mathbf{H}) = \max\limits_{x\in \mathbb{R}^n \setminus \{0\}} \dfrac{||x||'}{||x||}$.
\end{proposition}

\begin{proof}	
	\begin{align*}
		\dfrac{||x||'}{||x||} 	&= \dfrac{1}{||x||} \dfrac{d}{dt}\big(x^Tx\big)^{1/2} \\
										&= \dfrac{1}{2||x||^2} (x^T x' + (x')^T x) \\
										&=  \dfrac{1}{2||x||^2} (x^T \mathbf{A}x + x^T\textbf{A}^Tx) \\
										&= \dfrac{ x^T\mathbf{H}x}{||x||^2}
	\end{align*}

	This expression is a scale invariant function of $x$, so to maximize it, we only need to consider unit vectors. $$\max\limits_{||x||=1} x^T \mathbf{H} x$$
	%
	
	$\mathbf{H}$ is real symmetric, hence diagonalizable with an orthogonal change of basis. Let $\{\lambda_1, \lambda_2, \ldots \lambda_n\} = spec(\mathbf{H})$, in order from largest to smallest.
	\begin{align*}
		x^T\mathbf{H}x &= x^T(\mathbf{BDB}^T)x\\
		&= (x^T\mathbf{B})\mathbf{D}(\mathbf{B}^Tx)\\
		&= y^T \mathbf{D}y, ~\text{  where } y = \mathbf{B}^Tx \text{ is also unit.}\\
		&= \lambda_1 y_1^2 + \lambda_2 y_2^2 + \ldots +\lambda_n y_n^2.
	\end{align*}
The maximum of this expression over all $||y||=1$ is clearly $\lambda_1$, when $y= (y_1, \ldots , y_n) =(1, 0, \ldots, 0)$.

	
	%Since $\mathbf{H}$ is a symmetric matrix, it defines a quadratic form $Q(x) = x^T \mathbf{H} x$. The maximum value of a quadratic form restricted to the unit sphere is the largest eigenvalue of its matrix representation. \todo{cite}
	
\end{proof}

%If a linear system has positive reactivity, then there are arbitrarily small perturbations that will initially amplify, before eventually decaying to the sink. 



to do: relation between nonlinear and linearized systems

%connection to SVD?

% connection to Turing instability?





\subsection{Intensity of Attraction}

The definitions of asymptotic resilience and reactivity both require linearizing at a point attractor. In contrast, intensity of attraction, introduced by Katherine Meyer in her PhD thesis \cite{meyerMetricPropertiesAttractors2019}, measures resilience for any attractor, and captures metric information across its entire basin of attraction rather than replacing the nonlinear dynamics with a local approximation. 

Define attractor

Define control

Define reachable sets

Define intensity of attraction
