\section{Resilience Quantification}
\label{sec:resilience}

In this section, we review different approaches to quantifying resilience. 
%
First, we discuss the classic and most commonly used definition of resilience in theoretical ecology, also known as asymptotic resilience. Asymptotic resilience is defined to be the dominant eigenvalue of the Jacobian matrix at a stable equilibrium, and represents the long-term rate of return to that equilibrium after a small isolated perturbation away from it. 
%
Second, 


%
Finally, we discuss intensity of attraction, a quantification of resilience defined by Katherine Meyer in her PhD thesis \cite{meyerMetricPropertiesAttractors2019}. 


\subsection{Asymptotic Resilience}

 the dominant eigenvalue of the Jacobian at the stable rest point. The real part of this eigenvalue approximately governs the rate of return to equilibrium after a small perturbation to the system. Because local bifurcation is characterized by the real part of the dominant eigenvalue passing through zero, the system recovers slower when nearer to bifurcation. This definition of resilience (dominant eigenvalue of the Jacobian) is also

\subsection{Width of Basin of Attraction}

\subsection{Reactivity}

\subsection{Intensity of Attraction}
