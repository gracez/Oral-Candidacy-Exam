\section{Conclusion}
\label{sec:conclusion}




%Summarize and reiterate, etc. \todo{to do}



%Mention paper with the network model (I think?) where critical transition occurs with no lead warning.

Pressures exerted by modern day anthropogenic practices on the Earth are growing in magnitude and complexity, threatening physical, ecological, and social systems on all scales with unprecedented forms of change. Therefore, achieving a deeper understanding of how systems behave near critical transitions and how to identify early warning signs of impending critical transitions becomes an increasingly pressing goal. 

While generic early warning indicators hold powerful promise as a tool for advance prediction of critical transitions, they are not yet ready for practical use in management decisions, since their reliability is not sufficiently well understood. Rigorous mathematical theory can offer improved clarity about the conditions under which early warning indicators can be expected to occur. 

However, the current state of the theory typically relies on the long-term and local approximations associated with asymptotic resilience, which can be unrealistic. Instead, the role of transient behavior must be taken into account, and it will ultimately be important to develop a broader theory of transient dynamical indicators of critical transitions. 
This will require an adjustment to the definition of resilience underlying the theory - perhaps intensity of attraction offers a reasonable option. 